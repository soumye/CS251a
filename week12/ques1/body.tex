\section{Introduction}
Traditional long mulitplication method  takes a lot of time and are also generally difficult to carry out accurately and fastly in the head \cite{google}. For more details on the long division method go to : \url{http://mathworld.wolfram.com/LongMultiplication.html}. There are various methods given in vedas to multiply some specific type of 2-digit numbers very quickly. Two such examples are listed below. \cite{vedas} \cite{vedic}

\begin{table}[h]
	\begin{center}
		\begin{tabular}{|r | l|}
			\hline
			\textbf{TYPE} & \textbf{DESCRIPTION} \\
			\hline
			Type I & Two digit numbers with first digit same, and the last ones add up to 10. \\
			\hline
			Type II & Two digit numbers with first digit add up to 10, and the last ones are same. \\
			\hline
		\end{tabular}
		\caption{Table of Types}
	\end{center}
\end{table}

\section{Type I}
\subsection{Algorithm}
\label{Type I:algorithm}
	\begin{algorithm}
		\begin{algorithmic}
			\Procedure{Multiplication}{$a, b$}
				\State $c \gets a/10$ \Comment{First digit of the first number}	
				\State $d \gets a \; mod \; 10$ \Comment{Second digit of the first number}	
				\State $e \gets b/10$ \Comment{First digit of the second number}
				\State $f \gets b \; mod \; 10$ \Comment{Second digit of the second number}	
				\If{$c == e$ \& $d+f = 10$}
					\State $Part1 \gets c \times (c+1)$
					\State $Part2 \gets d \times f$
					\State $Product \gets Part1 \times 100 + Part2$ \Comment{Combining the two products}
					\State{\Return{$Product$}}
				\Else
					\State{\Return{Not of Type I}}
				\EndIf
			\EndProcedure
		\end{algorithmic}
	\end{algorithm}

